\section*{Introduction}
\label{sec:introduction}
\addcontentsline{toc}{section}{Introduction}

Spoofax is a language workbench developed by Delft University of
Technology over the course of several years. During those years it has
grown to be \textit{``a language workbench for efficient, agile
  development of textual domain-specific languages with
  state-of-the-art IDE support''}~\cite{Kats10a}.

When developing new domain-specific languages using Spoofax, the
ability for rapid prototyping using a read-eval-print loop (REPL)
would be very convenient. This report describes the initial research
phase for this project: extending Spoofax with a REPL that works with
all languages defined in Spoofax.

The first three sections of this research report give an overview of
the problem domain. \Cref{sec:spoofax} gives the scope of Spoofax and
each of its meta-languages. In \cref{sec:repl}, REPLs are explained in
more detail. After that \cref{sec:literate-programming} introduces
literate programming.

The remaining four sections concern the problem and the project. In
\cref{sec:problem-definition}, a short problem definition is given.
\Cref{sec:problem-analysis} then gives an analysis of this problem
definition in the context of the services that Spoofax already
provides, to clarify potential problems and their solutions when
creating a REPL that works with every language defined in Spoofax.
Next, in \cref{sec:requirement-analysis}, a list of design goals will
be formalized to guide the development of the deliverable. From these
design goals and the problem definition and analysis, a list of
requirements for the product will be compiled. Lastly,
\cref{sec:realisation-product} gives an account of the development
frameworks and tools to be used during the development of the
project.

%%% Local Variables:
%%% mode: latex
%%% TeX-master: "main"
%%% End:
