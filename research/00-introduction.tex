\section*{Introduction}
\label{sec:introduction}
\addcontentsline{toc}{section}{Introduction}

Spoofax is a language workbench developed by Delft University of Technology over
the course of several years. During those years it has grown to be \textit{"a
language workbench for efficient, agile development of textual domain-specific
languages with state-of-the-art IDE support"}~\cite{Kats10a}.

Since Spoofax is used to develop new domain-specific languages, the ability for
rapid prototyping using a read-eval-print loop would be very convenient.  The
first aim of this research report is to get a better overview of the scope of
Spoofax and each of its meta-languages. Afterwards, it will explore REPLs and
literate programming in order to clarify what functionality a good
implementation should have.

After having explored the domain of the project, a concise problem definition
will be given. This problem definition will then be analyzed in the context of
the services that Spoofax already provides to clarify potential problems and
their generic solutions to generate REPLs that work with languages defined in
Spoofax.

To conclude, the requirements for the product will be analyzed, as well as the
methods that will be used in order to realize these requirements.

%%% Local Variables:
%%% mode: latex
%%% TeX-master: "main"
%%% End:
