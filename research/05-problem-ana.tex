\section{Problem Analysis}
\label{sec:problem-analysis}
This section gives an analysis of the problem defined in the previous section.
More specifically, this section describes the problems that are expected and
that will have to be solved while incorporating a REPL
within the larger context of Spoofax.

A language can have many different language constructs. The REPL for Spoofax,
however, should not be constrained to any particular language constructs.
Rather, it should operate on any defined language. For this reason the
implementation and particularly the interaction with the user needs to be
carefully considered.

\subsubsection{Execution model of the REPL}
\label{ssec:execution-model-repl}
As illustrated in \cref{ssec:execution-model}, a major difference in the
execution model of a non-interactive interpreter and a REPL is concerned with
what an indivisable unit of execution is. The build process as implemented in
Eclipse uses a ``BuildInputBuilder'' to transform and build a (valid) program
as a whole. While this is quite a convenient way to quickly create a compiled
program; for a REPL to really offer an interactive experience to the user, the
REPL should be able to execute smaller units of work than a complete program.

An example implementation that solves this problem is the Java Shell (JShell)
REPL~\cite{jshell-repl} currently under development for OpenJDK. The OpenJDK
team introduces a concept called ``wrapping'' to achieve this: \textit{``if X
is an input that JShell accepts (as opposed to rejects with error) then there
is an A and B such that AXB is a valid program in the Java programming
language.''}. More specifically, this says that when a user types in an
expression like \texttt{int x = 2 + 2;}, JShell generates the surrounding class and
method body needed for this expression to be valid.

This technique cannot be directly applied to Spoofax, since the language
constructs needed to sucessfully ``wrap'' other constructs is not known in
advance. A first approach to solve this problem could be to require that the
language designer provides a program template in an esv file, in which they can
indicate the various types of language constructs that are valid from within
the REPL and where they would be inserted in this template. The REPL should
then be able to distinguish the language constructs defined in the esv file as
the user types them in. A template for a complete paplj program could look
like this:
\begin{lstlisting}
program
  $classes
run
  $expressions
\end{lstlisting}

The illustrated solution could later be extended to eliminate the need for
manual specification, by reusing the SDF syntax definitions for the language:
given an invalid partial program as typed in by the user, the REPL should then
try to determine whether this partial program is valid somewhere within the
parse tree resulting from the language definition. By doing a breadth-first
search, starting from the context-free start symbols of the language, the REPL
could prepend symbols from the syntax definition until the program is valid.
This suggestion can then be shown to the user, after which the resulting
program is executed.

When a compiler or an interpreter is invoked, the program is a complete and
static unit of source code that will not change during execution. In contrast,
when a REPL is invoked, often the user wants to alter or add code to previously
executed expressions. This means the REPL needs to form ``programs'' in memory
by combining all the entered expressions.  Thus the REPL should have a way of
combining previously entered expressions to form a program. JShell does this by
keeping the wrapped source and generated class files in memory and importing
previously defined classes~\cite{jshell-repl}. The concept of imports is
however very language specific and it cannot be assumed to exist.  Therefore,
another way of combining entered expressions into one program needs to be
found.

\subsubsection{Detecting unfinished expressions for multiline editing}
\label{sec:detect-unfin-expr}
To support multiline editing, the REPL should detect that the
expression or statement is unfinished when the user presses the
``Return'' key for the next line, instead of trying to parse and
execute it. An obvious part of Spoofax that is relevant for this
problem, is the syntax definition in SDF3 (see
\cref{sec:orgheadline1}).

\subsubsection{Language specific additional commands}
\label{sec:lang-spec-addit}
Another problem is when some additional command can be useful for a
REPL of a particular language, but would not make any sense within the
context of other languages. For example, some languages such as Python
allow for loading a file, which brings all of the definitions inside
of it into scope. An additional command to load a file could in that
case be useful, but would not make any sense in languages that have no
concept of definitions that can be imported.

This problem could be solved by the language designer by extending
their language with reflective capabilities. However, the language
designer might not want to extend the language with reflective
capabilities outside of the context of a REPL, for example when the
language is a DSL. It is clear therefore that a different approach
should be considered.

One possible solution is to allow for language specific configurations
that are loaded with the language definition of that language. This
can be done by extending the editor services discussed in
\cref{sec:editor-serv} with REPL commands: similar to menu actions,
where one can define menu button ``Run'' to run a program, one may
define a ``load-file'' command and bind it to an action.

\subsubsection{Redefining terms bound to names}
\label{sec:redef-cont-bound}
When the user is prototyping methods inside the REPL, they would
likely want to be able to redefine that method to be of a different
implementation. However, this poses a similar problem as in the
previous section: for some languages it may not be possible nor
desirable to do so outside of the context of a REPL. Thus requiring
the language designer to extend the language with such abilities is
again an inadequate solution.

One could propose the same solution as in the previous section, namely to
define an additional command for redefining a class or method. Another possible
and maybe more adequate solution for redefining a term bound to a name, is to
allow the user to give the name and the new term. The name can then be used to
find the old term, so that it can be replaced with the new one.

%%% Local Variables:
%%% mode: latex
%%% TeX-master: "main"
%%% End:
