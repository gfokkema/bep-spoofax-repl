\section{Problem Analysis}
\label{sec:problem-analysis}
This section gives an analysis of the problem defined in the previous
section. More specifically, this section describes the problems that
are expected and that will have to be solved during the implementation process
of the product. The first part of this section describes the process
of incorporating a REPL within the larger context of
Spoofax. The second part analyzes the part of the problem concerned
with literate programming. This part will be analyzed in less detail
than creating a REPL as it is unsure whether it will be present in
the final product.

\subsection{Creating a REPL that works with any language defined in Spoofax}
\label{ssec:incorp-repl-spoofax}

Every language can have many different language constructs. The REPL for
Spoofax, however, should not be constrained to any particular language
constructs like classes. Rather, it should operate on any defined language. For
this reason the implementation and particularly the interaction with the user
needs to be carefully considered. This subsections lists some of the problems
that are expected to arise from this.

\subsubsection{Detecting unfinished expressions for multiline editing}
\label{sec:detect-unfin-expr}
To support multiline editing, the REPL should detect that the
expression or statement is unfinished when the user presses the
``Return'' key for the next line, instead of trying to parse and
execute it. An obvious part of Spoofax that is relevant for this
problem, is the syntax definition in SDF3 (see
\cref{sec:orgheadline1}).

\subsubsection{Language specific additional commands}
\label{sec:lang-spec-addit}
Another problem is when some additional command can be useful for a
REPL of a particular language, but would not make any sense within the
context of other languages. For example, some languages such as Python
allow for loading a file, which brings all of the definitions inside
of it into scope. An additional command to load a file could in that
case be useful, but would not make any sense in languages that have no
concept of definitions that can be imported.

This problem could be solved by the language designer by extending
their language with reflective capabilities. However, the language
designer might not want to extend the language with reflective
capabilities outside of the context of a REPL, for example when the
language is a DSL. It is clear therefore that a different approach
should be considered.

One possible solution is to allow for language specific configurations
that are loaded with the language definition of that language. This
can be done by extending the editor services discussed in
\cref{sec:editor-serv} with REPL commands: similar to menu actions,
where one can define menu button ``Run'' to run a program, one may
define a ``load-file'' command and bind it to an action.

\subsubsection{Redefining terms bound to names}
\label{sec:redef-cont-bound}
When the user is prototyping methods inside the REPL, they would
likely want to be able to redefine that method to be of a different
implementation. However, this poses a similar problem as in the
previous section: for some languages it may not be possible nor
desirable to do so outside of the context of a REPL. Thus requiring
the language designer to extend the language with such abilities is
again an inadequate solution.

One could propose the same solution as in the previous section, namely
to define an additional command for redefining a class or
method. However, in this case this is not an adequate solution either,
since binding terms to names is a common feature among programming
languages. The fact that Spoofax comes with the NaBL DSL for
specifying name binding rules illustrates this. Name binding is in
fact the very feature which enables one to create abstractions.

A possible and more adequate solution for redefining a term bound to a
name, is to allow the user to give the name and the new term. The name
can then be used to find the old term, so that it can be replaced with
the new one.

\subsubsection{Partially replacing the AST of the current program}
When given an identifier to bind a new term to, the term in the AST
corresponding to the identifier has to be replaced with the new term.
This implies that the REPL needs to able parse single terms that might
not be valid programs and then update the AST of the program accordingly.
This requirement has certain effects when it comes to the build process
used in the REPL.

First of all, a REPL should always hold on to the AST representation of the
executing program, in order to find the old term in the current program.
Since Spoofax can just as well transform the AST back to original syntax,
it is possible to alter this identifier either directly as a Stratego term or
in the original syntax.

Secondly, the build process in Eclipse uses a ``BuildInputBuilder'' to
transform and build a (valid) program as a whole. While this is quite
a convenient way to quickly create a compiled program, it is desirable
that the REPL has a bigger influence on the way programs are built. A
REPL should therefore be able to execute individual Spoofax stages as
needed, instead of being limited to running the whole build process at
once.
%- wat er gebeurt als je een stuk AST transplanteert waardoor een method ineens invalid return type heeft

%%% Local Variables:
%%% mode: latex
%%% TeX-master: "main"
%%% End:
