\section{Problem Analysis}
\label{sec:problem-analysis}

To achieve a solid working REPL across all generated languages,
some Spoofax concepts have to be carefully considered.
Since every language that will be used in the REPL will be unique,
the REPL should not be constrained to language specific constructs like
classes or structs, but rather should operate on them in a generic way.
Therefore the implementation and particularly interaction with the user needs
to be thought of carefully, for example when implementing functionality to
alter the contents of namespaces.

\todo{FIXME: Just noting possible solutions we might want to elaborate}
We could add ...
\begin{itemize}
\item a special prefix for shell specific commands, and then...
\item a command to enumerate all possible namespaces as detected by spoofax
\item a command to switch to a namespace from this list
\item a command to list entries in the current namespace
\item expressions typed will be added to the selected namespace
\item all language agnostic, since we won't care what to call the namespace
\end{itemize}

\subsection{Incorporating REPL generation within Spoofax}
\label{ssec:architecture}

\subsubsection{The relevant parts of Spoofax}
\label{sec:which-parts-spoofax}

\subsubsection{The available APIs from Spoofax}
\label{sec:what-are-available}

\subsection{Plug-in development in Eclipse}
\label{ssec:eclipse-plugins}

%%% Local Variables:
%%% mode: latex
%%% TeX-master: "main"
%%% End:
