\section{Read-Eval-Print Loops}
\label{sec:repl}

Many programming languages come with an interactive environment. This
interactive environment is an interface to the programming language's execution
engine. One common form of such an environment is an interface in which
expressions in a programming language are typed by the user, after which the
results of that expression are printed back to the user. Such an environment is
called a Read-Eval-Print Loop (REPL), although many different names are known,
including but not limited to \textit{language shell},
\textit{command-line interpreter} or \textit{interactive interpreter}. There
are subtle differences between these names and the name REPL. These are,
however, mostly of semantic value. In this report the term REPL is chosen,
because it conveys the notion of such an interactive environment well.

\subsection{Origin of REPLs}
\label{repl-origin}

The Lisp programming language is one of the first programming
languages offering such functionality~\cite{Noyes92}. The name REPL comes from
the Lisp functions that implement it:

\begin{enumerate}
  \item The \texttt{read} function takes a user's input, which often is just one
    or several expressions as opposed to a complete compilation unit. It then
    parses this input and creates an abstract syntax tree (AST).
  \item The AST created in the previous step is then passed on to the
    \texttt{eval} function, which evaluates it.
  \item The result yielded by the previous function is then printed out to the
    user by the \texttt{print} function.
  \item After having printed the result, the environment needs to \texttt{loop}
    back to the read state.
\end{enumerate}

Assuming the individual functions listed previously exist, a REPL can be created
in a single line of code simply by combining the functions:

\begin{lstlisting}[language=lisp]
(loop (print (eval (read))))
\end{lstlisting}

Part of the semantic differences between the different names for an interactive
programming environment is Lisp's homoiconicity. In Lisp, the structure of
expressions is represented directly in a data structure, resulting in the
ability to infer a program's or data object's state simply by reading its
textual representation. In Lisp REPLs, therefore, arbitrary data objects yielded
from a previous expression can be used as input to the next expression. In
modern programming languages that do not belong to the Lisp familiy,
homoiconicity is an unusual feature. Interactive environments for these
languages therefore often require additional steps to read and evaluate
expressions. As said earlier, this difference is mostly semantic and therefore
this report uses the more general interpretation of the name REPL.

\subsection{Advantages of REPLs}
\label{ssec:repl-advantages}
REPLs offer several advantages that make them an often requested feature.
Because of their nature, REPLs provide the ability to program interactively.
Programming interactively has multiple applications.

When creating software solutions for an as of yet not well understood domain,
it is often not clear which data structures and algorithms are required. In such
cases, REPLs offer the ability to interactively develop and debug software
without having to apply the (oftentimes much slower) edit-compile-run-debug
development style. This kind of programming is called exploratory
programming~\cite{Fritzson86}. Related to this kind of exploration, REPLs also
provide a means for rapid prototyping and bottom up programming~\cite{Graham93}.

The explorative and interactive features of a REPL also make it an excellent
tool for programmers to learn a new programming language. REPLs are also
combined with what is called literate programming to offer notebooks or language
playgrounds, as discussed in~\cref{sec:literate-programming}.

\subsection{Functionality}
\label{sssec:repl-functionality}

Every programming language providing a REPL has its own set of functionality.
However, a core set of functionalities, shared between all REPL implementations,
can be identified. To reach this core set of features, well-known REPLs have
been investigated and their features have been compiled into a matrix as seen
in~\cref{table:feature-matrix}. These features are shortly discussed below.

\begin{table}[]
\centering
\begin{tabular}{lccccccccc}
                                  & \rot{Python} & \rot{R} & \rot{\shortstack[c]{Common\\Lisp}} & \rot{Haskell} & \rot{Swift} \\
\toprule
Executes single expressions       & \cmark       & \cmark  & \cmark                             & \cmark        & \cmark      \\
Executes statements               & \cmark       & \cmark  & \cmark                             & \cmark        & \cmark      \\
Input \& output history           & \cmark       & \cmark  & \cmark                             & \cmark        & \cmark      \\
Persistent input history          & \cmark       & \cmark  & \xmark                             & \cmark        & \cmark      \\
Multiline input editing           & \cmark       & \cmark  & \cmark                             & \cmark        & \cmark      \\
Redefining identifiers            & \cmark       & \cmark  & N/A                                & \cmark        & \cmark      \\
Error reporting                   & \cmark       & \cmark  & \cmark                             & \cmark        & \cmark      \\
Context-sensitive code completion & \cmark       & \xmark  & N/A                                & \xmark        & \cmark      \\
Help or documentation system      & \cmark       & \cmark  & \cmark                             & \xmark        & \xmark      \\
Additional commands to the REPL   & \xmark       & \xmark  & \cmark                             & \cmark        & \cmark      \\
Nested REPLs to enable debugging  & \xmark       & \xmark  & \cmark                             & \xmark        & \xmark      \\
\bottomrule
\end{tabular}
\caption{A feature comparison of several well-known REPLs}
\label{table:feature-matrix}
\end{table}

\paragraph{Input \& output history} REPLs keep a history of inputs and outputs,
such that previously entered expressions can be retrieved together with their
(evaluated) results. This can be with, say, a keyboard shortcut to retrieve
prevsiouly entered inputs and with variables bound to previously yielded values.

\paragraph{Persistent history} The input and output history as discussed
previously can be recorded into a file (either per-project or globally) to
enable a persistent history of input and output.

\paragraph{Multiline input editing} Constructs in a programming language are
not necessarily bound to one line; a method in Java for example often has the
following structure:
\begin{lstlisting}[language=java]
boolean isEven(int number) {
    return (number % 2) == 0;
}
\end{lstlisting}
One should be able to type these constructs in their habitual way. Therefore,
REPLs provide multiline input editors that recognise incomplete code and
promptly switch to a multiline environment when required.

A REPL is just another programming environment; just as in a regular editor or
IDE, therefore, keyboard shortcuts should be present to improve the workflow of
the programmer. This includes keyboard shortcuts to move over words, delete
words (both in forward and backward directions) and jump to the beginning or
end of the line.

\paragraph{Redefining identifiers} When using a REPL in an exploratory manner,
it is not uncommon to want to redefine an identifier's type or to completely
reimplement a method. In this way, a REPL can be different than its host
language, especially if the host language is a functional language that doesn't
allow variable's values to change once initiated.

\paragraph{Error reporting} A REPL should provide the same error reporting
capabilities as an IDE; preferably while typing input but at the very least it
should be able to report errors during the reading and evaluation phases.

\paragraph{Context-sensitive code completion} Context-sensitive code completion
is a helpful tool to provide an overview of the often many APIs a developer
works with, freeing them from having to remember everything. Note that this is
distinct from general tab-completion, which is not context-sensitive and is
offered by all the studied REPLs.

\paragraph{Help or documentation system} The exploratory nature of a REPL means
that one will often see new methods. It would be helpful to read one or several
lines of documentation in case it is required, so that the developer does not
have to switch contexts.

\paragraph{Additional commands to the REPL} Some REPLs offer additional
commands to inspect the environment or to control their behavior. These commands
are oftentimes not in the syntax of the language the REPL belongs to and are
highly diverse REPL implementations. A notable example of a REPL offering such
commands is Haskell's GHCi~\cite{GHCi-commands}.

\paragraph{Nested REPLs to enable debugging} A notable feature of (mostly) Lisp
REPLs is that in case of an error, a new REPL is spawned inside the context of
this error. This REPL then has additional commands (see the above feature) to
enable debugging and inspection of the error state. When the user has
resolved the error, the nested REPL exists and the user is returned to the
parent REPL. This can go to arbitrary depths.

%%% Local Variables:
%%% mode: latex
%%% TeX-master: "main"
%%% End:
