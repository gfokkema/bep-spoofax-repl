\section{Frameworks and Tools}
\label{sec:a-realisation-product}

The previous section analyzed the requirements for the product as set in
\cref{sec:a-requirement-analysis}. The purpose of this section is to discuss the
frameworks and tools used during the development of the product.

\subsection{Development frameworks}
\label{ssec:a-frameworks}

Spoofax is an already existing project. In order for the product to be as easily
integrated as possible, most of the tooling used by Spoofax will be reused. This
means that the Java programming language is to be used, together with Maven for
the build environment and JUnit for unit tests. On top of this, Spoofax uses a
few open source Java libraries as can be seen in the list below. It is most
likely that these will be used in the product as well.

The only exception to this is the fact that TravisCI is used as opposed to
Jenkins, because Jenkins is self-hosted whilst TravisCI is available for free
online.

\begin{itemize}
  \item Guice, a lightweight dependency injection framework;
  \item Guava, provides general utilities missing in the JDK;
  \item RxJava, a library for composing asynchronous and event-based programs
    using observable sequences;
  \item Apache VFS, a library for accessing various different file systems;
  \item EhCache, a standards-based cache that boosts performance;
  \item Apache's logging library.
\end{itemize}

If new libraries are to be used during the development of the project, it is
important that these are under a license compatible with the Apache license
under which Spoofax is licensed. Compatible licenses include, but are not
limited to, the LGPL, MIT, BSD and Apache licenses.

\subsection{Development tools}
\label{ssec:a-tools}

To manage the source code, the git version control system is used. Pull-based
development~\cite{Gousios14} is the paradigm chosen, in order for each member of
the team to work on their assigned tasks without interfering with each other.
To facilitate this, the GitHub platform is used because if offers an easy
interface to pull-based development: it automatically discovers the commits to
be merged, it facilitaties code review and discussion, new commits are
automatically added to the pull-request so that the discussion can continue and
GitHub automatically verifies whether commits can be merged. On top of this,
GitHub also provides an integrated issue tracker.

Furthermore, the tools FindBugs and PMD will be used to assist in delivering
quality code. Checkstyle will be used to guarantee a coherent style throughout
the code. These three static analysis tools will help ensure the design goal of
maintainability, as highlighted in \cref{ssec:a-goals}.

%%% Local Variables:
%%% mode: latex
%%% TeX-master: "main"
%%% End:
