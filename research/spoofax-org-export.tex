Spoofax is platform that allows for the completely \emph{declarative}
definition of a programming language and accompanying IDE
support~\cite{Kats10a}. The definition is done using high-level
\emph{meta-languages} for each aspect of the programming language.

To define a language declaratively means that one uses the
meta-languages to specify \emph{what} the properties of a language are, and
not \emph{how} these properties are implemented. For example, instead of
asking "How do I implement a tokenizer and parser for my language?",
one asks "What is the syntax of my language?". From such a description
in a meta-language, the tokenizer and parser can be derived, without
the designer of the language ever having to care about its
implementation.

This section goes over the parts that make up the specification of a
language, and for each part the relevant part of Spoofax is
given\footnote{This section follows the structure of the
language specification portion of the compiler construction course at
the TU Delft. The slides of this portion of the course can be found
here: \url{http://tudelft-in4303.github.io/lectures/specification/}.}. The
parts of a language specification are:
\begin{enumerate}
\item \hyperref[sec-syntax-def]{Syntax Definition}: Specifying the syntax of a language.
\item \hyperref[sec-static-analysis]{Static Semantics}: Describing the static analysis part of a
language: type checking, name binding and variable scoping.
\item \hyperref[sec-term-rewrite]{Term Rewriting and Program Transformation}: Rewriting ASTs to new
ASTs, for example to declare desugaring rules.
\item \hyperref[sec-dynamic-semantics]{Dynamic Semantics}: Defining what the language does upon execution.
\end{enumerate}

\subsection{Syntax Definition}
\label{sec-syntax-def}
The first part of the specification of a language is its syntax. The
syntax of a language is often specified by means of a \emph{lexical
grammar} and a \emph{context-free grammar}, as can be seen in the
specification of, for example, Standard ML~\cite{Milner97}. The
lexical grammar is most often defined using regular expression. It
defines the individual words made up of characters, such as
identifiers and numeric constants. The context-free grammar then
defines syntactically valid sentences made up of words.

\subsubsection{SDF3: syntax definition in Spoofax}
\label{sec:orgheadline1}
To specify a syntax definition declaratively in Spoofax, a DSL called
\emph{SDF3}~\cite{Vollebregt12} is used, the third generation of the
\emph{Syntax Definition Formalism} (SDF)~\cite{Heering89}. SDF3 uses
context-free grammer productions for the specification of both the
lexical syntax and the context-free syntax, a feature that was
introduced in SDF2~\cite{Visser97}.

The declarative nature of SDF3 allows for thinking in terms of the
structure (the \emph{what}), instead of in terms of parser algorithms (the
\emph{how}) as is the case with many current parsing
algorithms~\cite{Kats10b}. The syntax definition can be used to
make parsers which parse a textual representation to an AST, and
pretty-printers for mapping ASTs back to text. However, due to its
declarative nature, SDF3 is not limited to generating parsers and
pretty printers: it can also be used for error recovery
rules~\cite{deJonge12}, syntax highlighting rules and folding
rules for editors.

All of the other parts of Spoofax use the AST that is produced by a
parser generated from a SDF3 specification.
\subsection{Static Semantics}
\label{sec-static-analysis}
Static semantics refer to the meaning of what is a well-formed program
for a particular language~\cite{Milner97}. This imposes more
constraints than syntax definition, such as name binding, scoping
rules, and type checking. These cannot be specified by a syntax
definition alone, and are thus considered separately.
\subsubsection{Declarative static semantics specification in Spoofax}
\label{sec:orgheadline2}
In Spoofax, all the static semantics as well as the dynamic semantics
used to be specified with the Stratego transformation language (which
will be discussed in \hyperref[sec-term-rewrite]{Term Rewriting}). Nowadays, two high-level DSLs
exist for specifying static semantic declaratively: NaBL and TS. The
two DSLs can work together: for instance, the type of a variable can
be set in NaBL, so that TS can make assertions on it.
\subsubsection{NaBL: the Name Binding Language}
\label{sec:orgheadline3}
With NaBL (pronounced \emph{enable}), name binding and scoping rules can be
specified. Here is an example of the name binding and scoping rules
for a class, from the paplj language:
\begin{verbatim}
namespaces Program Class Field Method Variable
// ...
binding rules
  Class(c, _, _, _) :
    defines Class c of type ClassT(c)
    // Declare new scope
    scopes Field, Method, Variable
    implicitly defines Variable This() of type ClassT(c)

  Extends(c) :
    // Import namespaces from superclass
    imports Field, Method from Class c
\end{verbatim}
The first line declares the namespaces to
consider. Then for each node in the AST resulting from the parsing,
for example a \texttt{Class} node, the name binding and scoping rules can be
defined. In the example, each \texttt{Class} node declares a new scope for
its fields, methods and variables. It also implicitly defines the
\texttt{this} variable. The \texttt{Extends} node can then import the fields and
methods into its scope.

As can be seen from line 8, it can also associate type information
with names, to interplay with TS. The type annotations can also be
used for instance when desugaring or rewriting with Stratego (see \hyperref[sec-term-rewrite]{Term
Rewriting}).
\subsubsection{TS: the Type Specification Language}
\label{sec:orgheadline4}
Type checking can be done with the TS DSL. Again an example of the
paplj language:
\begin{verbatim}
type rules
  Class(c1, Extends(c2), _, _) :-
    where store ClassT(c1) <sub: ClassT(c2)

  x@This() : t
    where definition of x : t
// ...
type rules
  Add(e1, e2) : NumT()
    where e1 : NumT() else error "number expected" on e1
      and e2 : NumT() else error "number expected" on e2
\end{verbatim}
Rules can recursively set constraints on AST-nodes, such as the \texttt{Add}
node in the above example.

Again, in line 5, interplay can be seen between TS an NaBL. Here the
type of a variable can be accessed, which is set in the NaBL
specification (see previous section).
\subsection{Term Rewriting and Program Transformation}
\label{sec-term-rewrite}
Spoofax offers a high level declarative DSL called Stratego for program
transformation. Stratego operates on ASTs, and is the most general
part of Spoofax: it can be used for static semantics (name binding,
type checking), desugaring and for the dynamic semantics of a
language.

As the static semantics can now be done using NaBL and TS, and the
dynamic semantics with DynSem (see next section), Stratego can be used
to specify desugaring rules for a language.

Stratego is based on the notions of \emph{term rewrite rules} and so called
\emph{strategies}.
\subsubsection{Rewrite rules}
\label{sec:orgheadline5}
A rewrite rule is a transformation on a term, in which
the left-hand side allows for pattern matching and variable binding,
and the right hand side instantiates new replacement terms. An example
of a rewrite rule is given below.
\begin{verbatim}
rules
  desugar-let :
  	Let([], e) -> e

  desugar-let :
  	Let([b1, b2 | bs], e) -> Let([b1], Let([b2 | bs], e))
\end{verbatim}
This desugars a \texttt{let} expression with multiple bindings into multiple
nested \texttt{let} expressions each having just one binding.
\subsubsection{Strategies}
\label{sec:orgheadline6}
Strategies are used to select and apply term rewrite rules, to
construct the main algorithm of the program transformation. One can
use multiple combinators to compose rewrite rules and other
strategies. An example is given below:
\begin{verbatim}
strategies
  pre-desugar =
    innermost(desugar-let <+ desugar-do)

  post-desugar =
    innermost(desugar-do <+ desugar-get <+ desugar-set);
    resugar
\end{verbatim}
For example, \texttt{innermost} is a strategy to apply the strategy given as
parameter (a composition of rewrite rules) on the innermost AST node,
and repeats this until the strategy is no longer applicable\footnote{Is
this correct?}.
\subsection{Dynamic Semantics}
\label{sec-dynamic-semantics}
Dynamic semantics refers to how a program written in some language
behaves~\cite{Winskel93}. There are multiple approaches to
formally specify the dynamic semantics of a programming language (for
an extensive treatment, see~\cite{Winskel93}). For this section
only one sort of approach is relevant, namely \emph{rule-based operational}
\emph{semantics} (see~\cite{Plotkin04} for a historical account of this
approach).

\subsubsection{DynSem: a rule-based dynamic semantics}
\label{ssec-dynsem}
In Spoofax, the dynamic semantics of a language used to be specified
with Stratego. However, the Spoofax team has developed a more
high-level way to declare the dynamic semantics of a language, namely
a DSL called \emph{DynSem}~\cite{VerguNV15}. As with all DSLs in
Spoofax, DynSem offers a declarative approach to generate the
\emph{implementation} out of the \emph{specification}. Indeed, from a DynSem
specification of a language, an interpreter for that language can be
generated.

In DynSem, the dynamic semantics are specified by means of rules. To
show how rules can define the dynamic semantics of a language,
consider the classic example of the \(\beta\)-reduction, which defines
function application in the lambda calculus. The rule replaces all the
occurences of the parameter \(x\) with the argument \(e_2\), within the
expression \(e_1\):

\begin{equation}
(\lambda x.e_1) e_2 \rightarrow e_1[x := e_2]
\end{equation}

In a similar way, dynamic semantics can be specified in DynSem, in a
syntax very similar to the formal syntax used in the literature. Take
here the example of method calling in paplj:

\begin{verbatim}
rules
  // ...
  Call(o, m, vs: List(V)) --> v'
    where lookupMethod(o, m) --> Method(_, _, params, e);
          This o, Env bindVars(params, vs) |- e --> v'.
\end{verbatim}

The bottom line represents the rule of the method body, \(e\),
evaluating to the return value \(v'\), by binding the argument values to
the parameter in the environment and binding the \texttt{this} variable to
the object on which the method is called. Exactly how \(e\) evaluates to
\(v'\) is defined using other rules, which are left out in this example.
