\section{Problem Definition}
\label{sec:problem-definition}

The previous sections provided the necessary background knowledge of
the problem domain. Now that the background knowledge has been
explained, this section will discuss the problem definition of the
client.

Spoofax offers a wide variety of tools to develop DSLs and
accompanying IDE support. Precisely because Spoofax is concerned with
developing DSLs, rapid prototyping of syntax and grammar is a
convenient addition to the product. \Cref{sec:repl} showed that REPLs
provide this rapid prototyping ability for other languages like Lisp
and Python. Therefore, the client has expressed their interest in
a REPL that works with all language definitions created in
Spoofax. Such a REPL would aid both the developer of the language and
its end-user.

Since a lot of language services are already provided by Spoofax, the
REPL should try to hook into existing services as much as possible.
Besides exposing already existing Spoofax services in an interactive
shell, the product should also expose additional features geared
towards interactive exploration of programs in Spoofax defined
languages.  Examples include displaying and editing the contents of
current program context, saving and loading program context and
keeping a history of executed expressions.

When a REPL is realized, the product could be further extended to add
support for literate programming, allowing for rapid prototyping and
documentation at the same time~\cite{schulte2012}. This would allow
developers of new DSLs to document and explain their language in an
interactive manner that directly allows experimentation.

%%% Local Variables:
%%% mode: latex
%%% TeX-master: "main"
%%% End:
