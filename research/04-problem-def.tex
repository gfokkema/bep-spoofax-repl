\section{Problem Definition}
\label{sec:problem-definition}

In the first section the scope of Spoofax has been explored.
In the two subsequent sections REPL's and literate programming have been explained.
Given this information the problem definition can be adequately composed.

Spoofax offers a wide variety of tools to develop definitions of domain specific
languages. Given these definitions, Spoofax generates several DSL specific tools.
Precisely because Spoofax is concerned with developing DSLs, rapid prototyping
of syntax and grammar is a convenient addition to the product.
In section 2 REPLs were explored, and it was explained that these provide exactly
this rapid prototyping ability for other languages like LISP and Python.

The client has expressed their interest in a REPL that works with all languages
generated by Spoofax. Since a lot of language services are already provided
by Spoofax, the REPL should try to hook in to these services as much as possible.
Besides exposing already existing Spoofax functionality in an interactive shell,
our product should also expose additional features geared towards interactive
exploration of programs in the generated languages.
Examples include displaying and editing the current programming environment,
saving and loading context and a history of executed expressions.

When a REPL is realized, the product could be further extended to add support
for literate programming, allowing for rapid prototyping and documentation at
the same time~\cite{org-mode}. This would allow developers of new DSLs to
document and explain their language in an interactive manner that directly
allows experimentation.

%%% Local Variables:
%%% mode: latex
%%% TeX-master: "main"
%%% End:
