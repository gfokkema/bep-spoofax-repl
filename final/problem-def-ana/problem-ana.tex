\section{Problem Analysis}
\label{sec:problem-analysis}
This section describes the sub-problems that need to be solved while
incorporating a REPL within the larger context of Spoofax.

The general theme of the sub-problems that arise from the problem
defined in the previous section, is that the REPL should work
with any language defined with Spoofax. Languages differ in what
properties they have and how they are specified. For example, a
functional programming language with no mutability is specified
differently from an imperative and mutable programming
language. Finding a generic way to make the REPL work for all of these
languages, without any configuration from the part of the language
designer, is hard, and may even be impossible.

The solution to many of the problems is to identify which parts of the problem
can be solved generically, and which parts warrant configuration by the language
designer. When the REPL requires configuration by the language designer, enough
flexibility should be provided to do so. Each of
\cref{ssec:supp-diff-ways,ssec:lang-spec-addit,ssec:repl-spec-semant} discusses
a more concrete instance of this general problem.

\subsection{Supporting different execution pipelines}
\label{ssec:supp-diff-ways}
Different Spoofax languages have different ways of executing a program.
For example, some languages have an analysis step, whereas
others do not. Moreover, specifying the dynamic semantics of a language
can be done in multiple ways (see \cref{ssec:dynamic-semantics}): one
can use DynSem, Stratego, and some languages even implement a Java
backend\footnote{See IceDust:
  \url{https://github.com/MetaBorgCube/IceDust}}.

Furthermore, a REPL maintains an execution environment for the duration of the
REPL session, but precisely what an execution environment looks like differs
among languages.

\subsection{Language-specific commands}
\label{ssec:lang-spec-addit}
Another problem is when some additional command can be useful for a
REPL of a particular language, but would not make any sense within the
context of other languages. For example, some languages such as Python
allow for loading modules, which brings all of the definitions inside
of these modules into scope. An additional command to load a module
could in that case be useful, but would not make any sense in
languages that have no concept of definitions that can be imported.

This problem could be solved by the language designer by extending
their language with reflective capabilities. However, the language
designer might not want to extend the language with reflective
capabilities outside of the context of a REPL. It is therefore clear
that a different approach should be considered.

\subsection{REPL-specific semantics}
\label{ssec:repl-spec-semant}
Sometimes a property of some language stands in the way of the rapid
prototyping capabilities that are desirable for a REPL. For example, a
language like Haskell does not allow for redefining a function's
implementation in its normal semantics, whereas in the context of a
REPL this is usually what one wants.

This poses a similar problem as in the previous section: adding REPL-specific
semantics to the language should not affect the original semantics of the
language. Therefore, requiring the language designer to change the original
semantics of the language is an inadequate solution.

%%% Local Variables:
%%% mode: latex
%%% TeX-master: "../main"
%%% End:
