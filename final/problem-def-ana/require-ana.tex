\section{Requirements Analysis}
\label{sec:requirement-analysis}

Defining requirements upfront is important for several reasons: it is a contract
between the developers and the client, it guides the product development, it
enables the client to track the progress and finally it allows validation of the
deliverable. The requirements are listed in this section.

\subsection{Design goals}
\label{ssec:goals}

The client has expressed some high-level requirements, which are listed in this
subsection as design goals in order of priority. The design goals serve as an
important guideline when defining and implementing the invidual requirements. As
such, they can be considered the bounds within which all requirements must fit.

\paragraph{Language-agnostic} The REPL should not make any assumptions about its
host language: it must work with all languages defined with Spoofax. This also
means that a language-agnostic configuration interface must be provided, in
order to allow the language designer to configure and implement the
language-specific parts of the REPL. Note that this does not mean that the REPL
is expected to work out of the box for any language.

\paragraph{Minimal configuration} The REPL should require as little
configuration as possible. This means that the REPL allows the
language designer to reuse existing components of their language
definition in its configuration. Moreover, the language
designer should not have to configure areas of the REPL that can
instead be derived automatically.

\paragraph{Maintainability} Spoofax is an already existing, open source project
managed by several people. When the product is delivered, these people will take
over the maintainership. Therefore, it is important that the code is
maintainable. This means that the code should be well-documented, with low
coupling and high cohesion in the code's modules.

\paragraph{IDE-agnostic} Current stable releases of Spoofax integrate solely with
Eclipse. An effort is underway to make Spoofax IDE-agnostic and to
provide separate modules to tie Spoofax to IDEs. The REPL should keep this in
mind from the start and not tie itself to any IDE.

\paragraph{Performance} Spoofax's developers already focus quite a bit on
performance: both the generation and the use of the services are performant.
This should be no different for the REPL.

\paragraph{Modify Spoofax's existing codebase as little as possible} The product
should be an extension to Spoofax, which means the changes made to the existing
Spoofax codebase should be as small as possible. Preferably, the REPL service
should be a standalone module.

\subsection{Requirements}
\label{ssec:requirements}

Under the guidance of the design goals listed in the previous subsection, the
requirements compiled from the feature matrix discussed in \cref{sec:repl} and
meetings with the client are discussed below using the MoSCoW method.

\subsubsection{Must have}

Requirements listed under ``must have'' are of critical importance to the
usability and success of the deliverable. Without these, the product is not in a
workable state and is not likely to be accepted by the client. \emph{Must} can
also be considered an acronym for the Minimum Usable SubseT.

\paragraph{Interactive REPL} Per the definition of a REPL given in
\cref{sec:repl}, the REPL has to be interactive. It should evaluate single
statements and expressions typed in by the user and print the results back to
them.

\paragraph{Works with any language defined in Spoofax} Every service
in Spoofax operates from language definitions. It is evident that the
REPL should work with these definitions if it is to fit within
Spoofax. As for the parts of the REPL that do require configuration,
the REPL should provide a flexible and generic interface for that.

\paragraph{Input history} Users should be able to retrieve previously
typed expressions and statements to support the explorative and interactive
nature of a REPL.

\paragraph{Automatic binding of previously yielded values} In the same vein,
previously yielded results should be implicitly bound to automatically generated
identifiers to make their values available in future expressions.

\paragraph{Multiline input editing} Multiline input editing is a
crucial feature for user satisfaction. The user should be able to
enter expressions spanning multiple lines.

\paragraph{Error reporting} To support the interactivity of the REPL,
error reporting should be available in two ways: reporting errors
while typing an expression and reporting errors after entering the
expression. While typing a statement, on-the-fly error reporting
should indicate wrongly typed parts of the statement. After entering
the statement, any errors that occurred in the execution pipeline
should be displayed, whether they are parse errors, errors found
during static analysis or evaluation errors.

\paragraph{Syntax checked expressions} Supporting the above requirement, all
input should have its syntax checked on the fly.

\paragraph{Syntax highlighting} All expressions and statements (whether they are
currently being entered, displayed as previously entered input or displayed as
previously yieled results) should be syntax highlighted.

\paragraph{Integration with Eclipse} As a first implementation, the REPL should
integrate with Eclipse to provide an interface to the users.

\subsubsection{Should have}

Requirements listed under ``should-have'' are important, but not required for
a working product.

\paragraph{Ability to redefine identifiers} As explained in \cref{ssec:repl-functionality},
REPLs provide the ability to explore unknown problem domains. It is not a
far-fetched idea that a developer would want to change function implementations
or the types of certain variables. To support this, a REPL should allow users to
redefine identifiers. This might mean that the REPL needs different semantics
than the language it operates on, contrasting the design goal that the REPL
should be language-agnostic.

\paragraph{Environment inspection} The exploratory and interactive nature of
REPLs calls for the ability to inspect the current environment. This is to
replace the files with source code that a developer could otherwise inspect and
an initial step towards offering debugging features in the REPL.

\paragraph{Save and load REPL state} Often, developers want to save the current
state of their IDE and return to it later. As such, the REPL should allow their
state to be saved and restored.

\subsubsection{Could have}

Requirements listed under ``could-have'' are desirable, but not necessary.
These requirements often improve usability or customer satisfaction and are
included only if time permits.

\paragraph{Code completion} The multiline input editor should ideally function
just like an IDE's editor. Semantic code completion is not yet implemented in
Spoofax, so the REPL should provide syntactic code completion instead.

\paragraph{Hover over variables to see value, type and others} Another step
towards debugging would be the ability to hover variables with the mouse in
order to inspect their value, type and other known information. The difference
between this feature and the previously mentioned environment inspection is that
this feature works per variable.

\paragraph{Literate programming} As explained in
\cref{sec:literate-programming}, literate programming offers the advantage that
code and documentation go hand in hand. This allows developers of languages in
Spoofax to document and illustrate their language simultaneously with the
development: documentation and examples can never be outdated, because outdated
example code will halt the execution.

\paragraph{Integration with other IDEs (IntelliJ)} Generating Spoofax's editor
services for IntelliJ is currently a work in progress and it would be nice if
the REPL works in IntelliJ from the start.

\subsubsection{Won't have}

Requirements listed under ``won't-have'' are not to be implemented during this
project. They have been identified as possible features, but are outside of the
scope of this project and listed only as possible suggestions for further work.

\paragraph{GDB-style debugging and nested REPLs} Spoofax currently does not
generate any debugging features, which would be required for the REPL to offer
such functionality as a built-in feature. Writing a debugger is outside
the scope of this project and thus offering debugging capabilities inside
the REPL is something to consider for a later version.

\subsection{Minimal viable product}
\label{ssec:mvp}

The design goals and requirements listed in the previous subsections give a good
idea of what the deliverable should look like. It is possible, however, that due
to unforeseen problems, not all the listed requirements and design goals can be
met. It is important to therefore define a minimal subset of the design goals
and requirements that \emph{must} be present in the final deliverable: the
``must have'' requirements have to be implemented, whilst adhering to the first
two design goals.

%%% Local Variables:
%%% mode: latex
%%% TeX-master: "../main"
%%% End:
