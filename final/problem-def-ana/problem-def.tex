\section{Problem Definition}
\label{sec:problem-definition}

In this section, the problem of the client is defined. First the
general problem is given in the form of a question. After that the
general definition will be divided into multiple sub-definitions.

Spoofax offers a wide variety of tools to develop DSLs and accompanying IDE
support. Precisely because Spoofax is concerned with developing DSLs, rapid
prototyping of syntax and grammar is a convenient addition to the product.
\Cref{sec:repl} showed that REPLs provide precisely this rapid prototyping
ability.  Therefore, the client has expressed their interest in a REPL that
works with languages designed with Spoofax. Such a REPL would aid both the
developer of the language and its end-user. The main problem, thus, can be
stated as a question as follows: \textit{``How can a REPL be created that can be
used with languages defined with Spoofax?''}.

Spoofax already provides a lot of language services that can be reused. However,
this is not without issues, as Spoofax's infrastructure currently is not geared
towards interactive use. A subquestion, then, is: \textit{``How can Spoofax's
services be changed to support both interactive and non-interactive use?''}.

% TODO: Java-backend example requires a cite?
Additionally, there are currently two ways to define the dynamic semantics of a
language (see \cref{ssec:dynamic-semantics}); one can either use DynSem or Stratego.
Some languages even define their own interpreter writtin in Java. Therefore,
another subquestion is: \textit{``How can the REPL support multiple
interpreters?''}.

As listed in \cref{ssec:repl-functionality}, REPLs provide many features. Some
of these are not yet available in Spoofax. These features are critical to
exposing the interactive exploration of programs in languages defined with
Spoofax. Examples include displaying and editing the contents of the current
program's context, saving and loading this context and keeping a history of
executed expressions. This forms the third subquestion: \textit{``How do these
additional features fit into Spoofax's existing architecture?''}.

When a REPL is realized in time, the product could be further extended to add
support for literate programming, allowing for rapid prototyping and
documentation at the same time~\cite{schulte2012}. This would allow
developers of new DSLs to document and explain their language in an
interactive manner that directly allows experimentation.

%%% Local Variables:
%%% mode: latex
%%% TeX-master: "../main"
%%% End:
