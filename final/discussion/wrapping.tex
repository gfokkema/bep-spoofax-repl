\section{Wrapping}
\label{sec:wrapping}

During the research phase of the project the investigated solutions were
primarily focused on ``wrapping'' a partial program such that it becomes a
valid program as a whole, as the OpenJDK team has done for JShell (see
\cref{ssec:a-execution-model-repl}). A solution using templates for various
language constructs was proposed to keep this approach generic across all
languages supported by Spoofax.

However, as the project developed and DynSem was reimplemented significantly,
the implemented solution has changed quite drastically. Since the
parser employed by Spoofax uses integrated error recovery and can therefore
automatically complement incomplete programs, ``wrapping'' turned out to be
unnecessary. Instead, the ``ShellFacet'' introduced in \cref{sec:esv-extensions}
allows the language designer to configure several symbols which are
considered valid start symbols in the context of a REPL.

Since individual DynSem rules can be retrieved from the language specification,
these rules can then already be used to interpret the results of partial ASTs.
The custom start symbols together with the REPL-specific rules used to obtain
and pass execution contexts to DynSem then already results in a fully functional
REPL, without the need to wrap partial programs inside predefined templates.

%%% Local Variables:
%%% mode: latex
%%% TeX-master: "../main"
%%% End:
