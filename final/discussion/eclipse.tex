\section{Eclipse Multithreading Issues}
\label{sec:eclipse-multithread}

The implementation of the Eclipse frontend has been a source of exposing many
shortcomings in the initial designs. These shortcomings and how they have been
resolved are discussed in this section.

\subsection{Single- versus multithreading}

The initial design assumed that the frontends and the backend would run in the
same thread. For a console based REPL, this assumption holds and greatly
simplifies the design. However, this assumption does not hold when the backend
has a frontend using a multithreaded GUI toolkit. This
assumption resulted in two problems, which are listed separately in the next
sections. The solution and changes made to the design are then discussed
afterwards.

\subsection{Blocking- versus non-blocking input}

In a multi-threaded environment, asking graphical text entry widgets for the
entered text is rarely a blocking process. The REPL backend at the time,
however, assumed that getting the user's input was always a blocking operation.
Therefore, when a conceptual Eclipse frontend was made, the REPL spun into an
infinite loop trying to execute empty expressions.

\subsection{Evaluating on the UI thread}

As explained in \cref{ssec:eclipse-plugin}, multi-threaded graphical user
interface toolkits often use multiple threads. One of these threads is
designated the ``UI thread'', or user interface thread. This thread is
responsible for processing events (such as mouse clicks) and updating the
graphical representation of the widgets. All tasks that perform long running
calculations are supposed to be run in a background thread, such that the UI
thread is free to process incoming events. Instead, if a long running
computation is run in the UI thread, the widgets on the screen stop responding
to the user and the program appears to be in a frozen state. This is exactly
what happened when the backend assumed to be run in the same thread as the
frontend: whilst the backend was evaluation expressions, Eclipse appeared to be
frozen due to this evaluation taking place in the UI thread from which the
execution was started. This issue would be worse in case blocking input
were to be used.

\subsection{Accustoming to multi-threaded frontends}

As indicated in the previous sections, the only solution to these problems is to
allow multi-threaded frontends. This meant that the assumption of every
operation running in the same thread no longer held. As a result, several
changes had to be made:

\begin{enumerate}
  \item Initially, a \texttt{Repl} class was present in the backend to
  centralize a REPL implementation. This class made too many assumptions on
  behalf of the frontends, among which blocking user input and the entire main
  loop of a REPL. The alternative is the \texttt{IRepl} interface (see
  \cref{sec:overview}), which defines the only method each frontend has in
  common: the \texttt{eval} method to evaluate input. It is now entirely up to
  the frontend on how to implement the read, print and loop steps, allowing for
  much more freedom.
  \item	The initial design provided ``hooks'' to the frontend to process results or
  errors. A frontend would register itself to process certain hooks, after which
  the registered method would be called. The registered methods would be called
  immediately after the evaluation was done, and on the same thread. In the
  conceptual Eclipse frontend, these hooks updated the widgets running in the UI
  thread. This then led to widgets being updated from a thread other than the UI
  thread. The solution, discussed in \cref{sec:visitor}, was to return the
  result to the frontend, allowing it to process the result whenever and however
  it needs.
\end{enumerate}

Not only did these changes resolve the threading issues, they also made for a
cleaner architecture overall. An additional advantage is that a much wider scala
of possible frontends is now supported.

%%% Local Variables:
%%% mode: latex
%%% TeX-master: "../main"
%%% End:
