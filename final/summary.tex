\chapter{Summary}
\label{cha:summary}

\textit{``The Spoofax Language Workbench supports the definition of all aspects
of textual languages using high-level, declarative meta-languages. From a
language definition using these meta-languages, Spoofax generates full-featured
Eclipse and IntelliJ editor plugins, as well as a command-line interface.''}

A feature that Spoofax is lacking is a Read-Eval-Print Loop (REPL) service. A
REPL is an interactive programming environment that takes single expressions,
evaluates them and prints the result(s). REPLs facilitate exploratory
programming and debugging and are therefore a nice addition to Spoofax.

This report comprises the end of the TU Delft Computer Science Bachelor Project;
the final, compulsory project at the Delft University of Technology to attain
the Bachelor of Science in Computer Science.

The purpose of this report is to describe the project as it was carried out over
the course of ten weeks. Not only does it describe the process, it also aims to
inform the reader about the problem, the design and the implementation of the
solution to this problem. Finally, recommendations are made to the client to
improve the delivered product.

To successfully complete the project, extensive knowledge needed to be gained of
both the conceptual ideas behind programming language implementations and of the
Spoofax API implementing these concepts.

Once the required knowledge was attained, two worthwhile contributions have been
made. First and foremost, a properly functioning REPL has been created,
comparable in features to those of popular programming languages such as Python
and Haskell. Secondly, significant changes have been contributed to DynSem and
have been integrated into the main repository.

Despite having faced significant challenges during the start of the project, the
most important goals as set forth in the problem description have been achieved.
Therefore, the project team is still quite satisfied with the end result: while
not all requested features have been implemented, it has been shown that it is
possible to create a functioning REPL for any language defined in Spoofax,
requiring only a minimal configuration in addition to the language definition.

%%% Local Variables:
%%% mode: latex
%%% TeX-master: "main"
%%% End:
