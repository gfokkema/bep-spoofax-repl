\chapter{Conclusion}
\label{cha:conclusion}

Over the course of the last quarter the Spoofax Language Workbench has been
explored and a Read-Eval-Print Loop (REPL) has been created that operates with
any language defined in the Spoofax.

\Cref{cha:introduction} gave the required background knowledge needed to
understand the problem domain. \Cref{cha:probl-defin-analys} framed the problem
definition in the context of this background. \Cref{cha:design} and
\cref{cha:implementation} discussed the design and the implementation of the
final product. \Cref{cha:evaluation} evaluated both the final product and the
process by which it came to be. \Cref{cha:discussion} reflected upon the
project, after which \cref{cha:recommendations} made recommendations to improve
the final product.

To successfully complete the project, extensive knowledge needed to be gained of
both the conceptual ideas behind programming language implementations and of the
Spoofax API implementing these concepts.

Once the required knowledge was attained, two worthwhile contributions have been
made. First and foremost, a properly functioning REPL has been created,
comparable in features to those of popular programming languages such as Python
and Haskell. Secondly, significant changes have been contributed to DynSem and
have been integrated into the main repository.

Despite having faced significant challenges during the start of the project, the
most important goals as set forth in the problem description have been achieved.
Therefore, the project team is still quite satisfied with the end result: while
not all requested features have been implemented, it has been shown that it is
possible to create a functioning REPL for any language defined in Spoofax,
requiring only a minimal configuration in addition to the language definition.

%%% Local Variables:
%%% mode: latex
%%% TeX-master: "main"
%%% End:
