\chapter{Conclusion}
\label{cha:conclusion}

Over the course of the last quarter our project team has explored the Spoofax
language workbench and created a read-eval-print-loop that operates with any
language defined in the Spoofax workbench.

During the research phase of the project it became apparent that the project
team needed to gain quite extensive knowledge about both the conceptual ideas
of language transformation and the Spoofax API implementing these concepts
before a successful product would become feasible. Due to this knowledge gap,
development progress has unfortunately been slower than anticipated for several
features.

Furthermore the Spoofax project is developing in a fast pace as it is preparing
for their new 2.0 release. A consequence of this has been the complete
replacement of the interpreter backend. This replacement meant that the
interaction of our REPL with the interpreter needed to be completely rewritten
halfway during the project.

After the project team gained the required knowledge and learnt how the new
interpreter backend was designed with some help from the client, ultimately
significant changes have been contributed upstream. This has resulted in the
ability to facilitate a real, functioning REPL, executing within a modifiable
context.

Despite the significant challenges the project team faced during the start of
the project, most goals set forth in the problem description have been
achieved. Therefore the project team is still quite satisfied with the end
result. While not all requested features have been implemented, we have shown
that it is possible to create a functioning REPL for any language defined in
Spoofax, requiring only minimal changes to the language definition.

%%% Local Variables:
%%% mode: latex
%%% TeX-master: "main"
%%% End:
