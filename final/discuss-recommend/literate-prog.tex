\subsection{Literate Programming}
\label{sec:literate-programming}
Just as with REPLs, the concept of literate programming is implemented in
various forms under various names. Therefore this section starts with an
explanation of what literate programming is exactly based on a few
implementations. Afterwards, the IPython implemention of literate programming is
explored in more detail.

Another improvement is to extend the product with support for literate
programming, as discussed in \cref{sec:literate-programming}. This would allow 
for rapid prototyping and documentation at the same time~\cite{schulte2012},
allowing developers of new DSLs to document and explain their language in an
interactive manner that directly allows experimentation.

%%% Local Variables:
%%% mode: latex
%%% TeX-master: "../main"
%%% End:
