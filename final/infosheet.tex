\chapter{Project Infosheet}
\label{cha:infosheet}
\addtocontents{toc}{\protect\setcounter{tocdepth}{0}}
\newpage

\noindent
\textbf{Title of the project:} Spoofax Shell\\
\textbf{Name of the client organization:} SLDE group, Delft University of Technology\\
\textbf{Date of the final presentation:} July 1st, 2016\\
\textbf{Description:} Create a command-line shell interface for the Spoofax
Language Workbench\\

The TU Delft SLDE group conducts research into concepts and techniques for
programming language design and implementation. The flagship product of the
group is the Spoofax Language Workbench.

The deliverable for this project was to create a REPL for the Spoofax Language
Workbench. The challenge was to figure out how such a REPL fits within the larger
context of Spoofax, and how to create a generic REPL that does not make assumptions
on its host language. In order to integrate the REPL, the students not only had
to learn what Spoofax is and how it works, but also how it is build and how the
different parts fit together. The existing architecture of Spoofax directed the
architecture of the product.

The development process was managed with the Scrum software development
framework. The source code was managed using Git, using the pull-based
development methodology.

The largest unexpected challenge faced during the project is the rewrite DynSem
halfway through the project. This new version of DynSem did not expose the
required functionality for a REPL. Collaborating with its developer, these
features were implemented over the course of two weeks.

The delivered product consists of a backend, providing a means of parsing,
analyzing and evaluating expressions using Spoofax's services. Two frontends
were delivered: one running in the console and the other inside the Eclipse IDE.
Both the frontends and the backend are tested with unit- and integration tests.

TODO: Outlook: Describe the outlook of the product. Did the team make recommendations to the 
client? (if so, briefly summarize) Will the product be used?\\

\noindent\textbf{Members of the project team}\\
\textbf{Gerlof Fokkema'} interests are Operatings Systems, networking and
programming languages. Gerlof primarily worked on the backend, specifically the
various Spoofax commands available to the user from within the REPL.\\
\textbf{Jente Hidskes'} interests are Operating Systems and programming
languages and the collaboration between the two. Jente mainly worked
on the frontends of the REPL and the communication with the backend.\\
\textbf{Skip Lentz's} interests lie in the field of programming languages,
both object-oriented (Smalltalk) and functional (Haskell), which provide
novel and more efficient ways for reasoning about program logic. Skip mainly
worked on the backend, specifically the inter-operability with DynSem's
generated interpreters.\\

\noindent All team members contributed to the reports and the final
presentation.\\

\noindent\textbf{Client:} Eelco Visser, SLDE group TU Delft\\
\textbf{Coach:} Hendrik van Antwerpen, SLDE group TU Delft\\
\textbf{Contact:}
\href{mailto:gerlof.fokkema@gmail.com}{gerlof.fokkema@gmail.com},
\href{mailto:hjdskes@gmail.com}{hjdskes@gmail.com},
\href{mailto:skippetie@gmail.com}{skippetie@gmail.com}\\

\noindent The final report for this project can be found at: \url{http://repository.tudelft.nl}.

%%% Local Variables:
%%% mode: latex
%%% TeX-master: "main"
%%% End:
