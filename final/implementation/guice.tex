\section{Dependency Injection}
\label{sec:injection}

The design of Spoofax relies quite heavily on dependency injection by the Guice
framework.  To achieve as much interopability as possible, the final product
uses dependency injection in the same spirit as Spoofax. Guice has had a
substantial impact on the design and implementation of the final product, since
Guice nearly eliminates the need to create factories for complex datatypes.

Guice requires the programmer to bind implementations of dependencies in its
module classes, thereby separating behavior and dependency resolution.  Complex
datatypes can then accept their dependencies as constructor arguments, allowing
Guice to resolve and inject an instance of a bound type.

In the final product dependency injection is also used to supply several classes
with a default configuration.  An example is the default list of commands
available to a user as described in \cref{sec:commands}.  The list of commands
is created from a Guice module by instantiating a \texttt{MapBinder} with
predefined bound commands. Child modules can then append extra entries to the
\texttt{MapBinder}, which makes them directly available to the existing
architecture.

To conclude, the described features have made it easier to create a modular
product, mostly centered around smaller interfaces interacting with eachother.
All these interfaces can easily be bound to new implementations, thereby
extending the functionality of the REPL.
