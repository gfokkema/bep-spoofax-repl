\chapter{Introduction}
\label{cha:introduction}

General purpose programming languages such as Java and C have historically been
the most prevalent in the field of computer science. Nowadays a wide variety of
domain specific languages is also available, each of which is designed to
efficiently solve particular domain specific issues. \textit{``Domain-specific
languages (DSLs) provide high expressive power focused on a particular problem
domain. They provide linguistic abstractions and specialized syntax
specifically designed for a domain, allowing developers to avoid boilerplate
code and low-level implementation details''}~\cite{Kats10a}. Well known
examples of such DSLs include HTML for creating websites and SQL for dealing
with relational data.

At the Delft University of Technology, the Software Language Design and
Engineering group researches and develops domain specific languages. To
facilitate prototyping and creation of DSLs, the Spoofax Language Workbench was
developed and first published in 2010~\cite{Kats10a}. Since then Spoofax has had
several releases and will soon be arriving at version number 2.0. The
description below is taken from \href{http://spoofax.org}{the Spoofax website}:
\textit{``The Spoofax Language Workbench supports the definition of all aspects
of textual languages using high-level, declarative meta-languages. From a
language definition using these meta-languages, Spoofax generates full-featured
Eclipse and IntelliJ editor plugins, as well as a command-line interface.''}

A feature that Spoofax is lacking is a Read-Eval-Print Loop (REPL) service. A
REPL is an interactive programming environment that takes single expressions,
evaluates them and prints the result(s). REPLs facilitate exploratory
programming and debugging and are therefore a nice addition to Spoofax.

This report details how such a REPL has been implemented over the course of the
TU Delft Computer Science Bachelor Project. First of all an overview of the
required background needed to understand the problem domain is given in
\cref{cha:background}. In \cref{cha:probl-defin-analys} the problem definition
is then framed in the context of this background. The design and implementation
of the final product are discussed in \cref{cha:design} and
\cref{cha:implementation}. After this \cref{cha:evaluation} evaluates both the
final product and the process leading up the final product. The whole
project is reflected upon in the discussion in \cref{cha:discussion}, after
which some recommendations regarding Spoofax and the final product will be made
to the client in \cref{cha:recommendations}. \Cref{cha:conclusion} then
concludes this report.

%%% Local Variables:
%%% mode: latex
%%% TeX-master: "main"
%%% End:
