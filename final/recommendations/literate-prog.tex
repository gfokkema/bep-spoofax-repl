The client expressed their interest in literate programming functionality for
Spoofax, as explained in \cref{sec:literate-programming}. Implementing an
IPython kernel on top of the core shell module should not take much effort,
however due to the problems discussed earlier priority was given to a solid
REPL and an Eclipse plugin. Therefore, the product currently does not deliver an
implementation of literate programming.

IPython kernels provide a solid framework for literate programming that has
proven itself with many different languages%
\footnote{See: \url{https://github.com/ipython/ipython/wiki/IPython-kernels-for-other-languages}}.
An IPython kernel is a daemon with several network sockets
representing the input and output streams of the frontend application. Several
message types are sent over these sockets in JSON format, such as requests to
evaluate a certain string or requests regarding the kernel
state\footnote{See: \url{https://jupyter-client.readthedocs.io/en/latest/kernels.html}}.

Since the invoker class from the backend accepts any string given to it and then
resolves the command itself, implementing an IPython kernel should be
straightforward. Most of the required work would be in creating an adequate
messaging framework capable of understanding all defined JSON messages. When the
messaging framework is in place, user input can be sent directly to the invoker
after which the results can be returned via network sockets by visiting the
\texttt{IResult} interface, similar to any other client implementation.

All functionality offered by Jupyter notebooks as shown in \cref{fig:ipython}
will be available to the Spoofax REPL at once by implementing an IPython kernel
(if the language supports these features). This directly results in the
ability to live edit code interspersed with documentation, while also allowing
more complex graphical elements.

%%% Local Variables:
%%% mode: latex
%%% TeX-master: "../main"
%%% End:
