\subsection{More extensive history functionality}

The current input history implementation provides a means of iterating the
previously entered expressions in a linear way: a user can scroll back and forth
through the old entries. It would be nice if the history was searchable, too.
When implementing this, one could turn to existing command-line shells for
inspiration. The GNU readline library, for example, has two ways of searching
through the input history: via a keyboard shortcut, which when pressed allows
the user to enter a word that they remember is in the entry they are looking
for, or via an always-on setting that allows the user to enter the beginning of
the expression which will in turn filter the linear iteration over the
history to only those entries starting with the entered input.

Another limitation of the current history implementation is the fact that it is
recorded per session. This means that when a user uses several languages in the
same session, the history will contain entries from both languages. Instead,
history should perhaps be kept not only per session, but also per language.

Related to the above, is implementing persistent history. The Eclipse frontend
currently only provides volatile history, which means that history is thrown
away when Eclipse is closed. The console plugin does provide persistent history,
but this implementation also leaves things to be desired: all history is saved
into the same file, resulting again in a mixed history. The recommendation that
is made here is to provide the ability to have separate, per language
persistent history. If the frontend is an IDE, these files can perhaps be saved
inside the language project's directory structure.

Improving the history implementation should result in a smoother and
faster way to interact with the REPL, enhancing the explorative nature.

\subsection{Analysis in context}

%%% Local Variables:
%%% mode: latex
%%% TeX-master: "../main"
%%% End:
