The Eclipse frontend that is delivered with the product has been discussed in
\cref{ssec:eclipse-plugin}. The plugin provides interaction with the REPL
backend, as discussed in \cref{cha:design}.

The delivered Eclipse frontend is just that: a frontend to the REPL backend.
Integration with Eclipse and Spoofax Eclipse are not yet present. This
section provides several recommendations to better integrate this frontend
into Eclipse and Spoofax Eclipse.

\subsection{Building languages and projects}

To start a REPL session in the delivered project, the language designer has to
build the language, start the REPL and issue a \texttt{``:load''} command to
load the language. It is desirable if all this could be done automatically at
the press of a button. Such a \textit{``Run in REPL''} button can be added to the
``Spoofax (meta)'' submenu and inside the project and package explorer context
menus.

When the REPL gains the ability to load existing files, such a button can also
be added to the ``Spoofax'' submenu to assist a language user.

An obstacle in implementing this feature is the classpath issue as described in
\cref{sec:classpath}.

\subsection{Interaction between Spoofax Eclipse and the Eclipse plugin}

The REPL backend supports loading individual files to bring the definitions
inside these files into scope. This functionality, however, is not yet provided
by the Eclipse plugin by means of menu items or buttons, nor is there a deeper integration
between the Eclipse plugin and Spoofax Eclipse.

Deeper integration improves the interaction between Spoofax Eclipse and the
Eclipse frontend. For example, hovering over variables could indicate where they are defined, or
changes in a currently loaded file could be automatically picked up. Changes in
the other direction are interesting, too: when a function definition is overridden
inside the REPL, an option could be provided to apply this change to its
originating file as well. This could even be extended to write (parts of) the current
environment to a new file.

\subsection{Improving the UI}

The user interface currently offered by the Eclipse frontend does not resemble a
typical REPL: the input and output views are separated. To improve the user
experience, these two views could be merged into one. This would require the use
of a widget or text representation that can be \textit{``semi-read-only''}, as
the previously entered text and results should not be editable, while the
current input should be.

%%% Local Variables:
%%% mode: latex
%%% TeX-master: "../main"
%%% End:
