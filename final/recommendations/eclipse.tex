The Eclipse frontend that is delivered with the product has been discussed in
\cref{ssec:eclipse-plugin}. The plugin provides interaction with the REPL
backend, as discussed in \cref{cha:design}.

The delivered Eclipse frontend is just that: a frontend to the REPL backend.
Integration with Eclipse and Spoofax Eclipse are not yet present. This
section provides several recommendations to better integrate this frontend
into Eclipse and Spoofax Eclipse.

\subsection{Building languages and projects}

To start a REPL session in the delivered project, the language designer has to
build the language, start the REPL and issue a \texttt{``:load''} command to
load the language. It is desirable if all this could be done automatically at
the press of a button. Such a \textit{``Run in REPL''} button can be added to the
``Spoofax (meta)'' submenu and inside the project and package explorer context
menus.

When the REPL gains the ability to load existing files, such a button can also
be added to the ``Spoofax'' submenu to assist a language user.

An obstacle in implementing this feature is the classpath issue as described in
\cref{sec:classpath}.

\subsection{Back and forth interaction between Eclipse and the REPL}

A feature the REPL does not yet support is loading existing files into the
evalution context (see \cref{ssec:discuss-repl}). When it does, this interaction
could provide deeper integration between the REPL and Spoofax Eclipse. For
example, hovering over variables could indicate from where they originate, or
changes in a currently loaded file could be automatically picked up. Changes in
the other direction are interesting, too: when a loaded function is overridden
inside the REPL, an option could be provided to apply this change to the loaded
file as well.

\subsection{Improving the UI of the Eclipse REPL}

The user interface currenty offered by the Eclipse frontend does not resemble a
typical REPL: the input and output views are separated. To improve the user
experience, these two views could be merged into one. This would require the use
of a widget or text representation that can be \textit{``semi-read-only''}, as
the previously entered text and results should not be editable, while the
current input should be.

%%% Local Variables:
%%% mode: latex
%%% TeX-master: "../main"
%%% End:
