\chapter{Evaluation}
\label{cha:evaluation}

Two evaluations are made in this chapter. The first section of this chapter,
\cref{sec:product-evaluation}, evaluates the product. This is done firstly by
looking at the requirements and secondly by looking at the design goals, both as
described in \cref{sec:requirement-analysis}. The end of this section concludes
with an evaluation of the product as a whole in \cref{ssec:eval-product}. The
second section of this chapter, \cref{sec:process-evaluation}, evaluates the
process by which the product came to be. It discusses the development
methodologies used and the feedback as received by the Software Improvement
Group.

\section{Product Evaluation}
\label{sec:product-evaluation}

\Cref{sec:requirement-analysis} describes the requirements and the design goals
that were agreed upon with the client at the start of the project. This section
evaluates how these are met by the product. \Cref{ssec:eval-requirements}
discusses each requirement and decides whether or not it has been met and how.
\Cref{ssec:eval-design-goals} does the same for the design goals.  Finally, this
section concludes with a short discussion to see if the minimal viable product
as defined in \cref{ssec:mvp} has been met.

\subsection{Evaluation of the Requirements}
\label{ssec:eval-requirements}

This subsection discusses all the requirements as set out in
\cref{ssec:requirements}. All the requirements are listed in
\cref{table:requirements}. Per requirement, the table lists whether it has been
met, including an explanation as to how it has been met or why it has not been
met.

% TODO: update history when PR is accepted.
\begin{table}[]
\centering
\begin{tabular}{|l||l||l|}
\hline
\textbf{Requirement}                                                                         & \textbf{Met?} & \textbf{Explanation} \\ \hline
Interactive REPL                                                                             & Yes           & \begin{tabular}[c]{@{}l@{}}Two interactive REPLs have been created:\\ one for the console and another in Eclipse.\end{tabular}\\ \hline
\begin{tabular}[c]{@{}l@{}}Works with any language\\ defined in Spoofax\end{tabular}         & Yes           & \begin{tabular}[c]{@{}l@{}}The backend retrieves the information it\\ needs from Spoofax's services. A language\\ developer is able to define REPL-specific\\ configuration in an esv file, as well.\end{tabular} \\ \hline
Input history                                                                                & Yes           & \begin{tabular}[c]{@{}l@{}}An interface has been created in the\\ backend, allowing frontends to implement\\ history as they see fit. Both the console-\\ and the Eclipse frontend implement this\\ interface.\end{tabular} \\ \hline
\begin{tabular}[c]{@{}l@{}}Automatic binding of \\ previously yielded values\end{tabular}    & Yes           & \begin{tabular}[c]{@{}l@{}}The language designer can specify these\\ semantics in DynSem.\end{tabular} \\ \hline
Multiline input editing                                                                      & Yes           & \begin{tabular}[c]{@{}l@{}}The way in which input is given is not\\ enforced by the backend. However, both\\ the console- and the Eclipse frontend\\ implement multiline input editing.\end{tabular}\\ \hline
Error reporting                                                                              & Yes           & \begin{tabular}[c]{@{}l@{}}Both the console- and the Eclipse frontend\\ print errors to the user. The characters to\\ which these errors belong are highlighted\\ in red. Error reporting is not yet enabled\\ during typing.\end{tabular}\\ \hline
Syntax-checked expressions                                                                   & Yes           & \begin{tabular}[c]{@{}l@{}}The backend parses the input with the\\ services provided by Spoofax. As such,\\ syntax-checked input was given for free.\\ However, this feedback is not yet enabled\\ during typing.\end{tabular}\\ \hline
Syntax highlighting                                                                          & No            & \begin{tabular}[c]{@{}l@{}}The Spoofax services to provide syntax\\ highlighting are difficult to fit into the\\ architecture of the REPL. For this reason,\\ syntax highlighting is not yet completed.\end{tabular}\\ \hline
Integration with Eclipse                                                                     & Yes           & \begin{tabular}[c]{@{}l@{}}A REPL has been implemented in Eclipse,\\ allowing the user to give input and see output.\end{tabular}\\ \hhline{|=|=|=|}
Ability to redefine identifiers                                                              & Yes           & \begin{tabular}[c]{@{}l@{}}The language designer can speficy these\\ semantics in DynSem.\end{tabular} \\ \hline
Environment inspection                                                                       & No            & \begin{tabular}[c]{@{}l@{}}In the current DynSem implementation, this\\ is impossible to do because the environment\\ is captured in an object of type \texttt{Object}\\ and therefore unprintable.\end{tabular} \\ \hline
Save and load REPL state                                                                     & No            & \begin{tabular}[c]{@{}l@{}}Due to limitations in time, this feature has\\ not been implemented.\end{tabular}\\ \hhline{|=|=|=|}
Syntactic code completion                                                                    & No            & \begin{tabular}[c]{@{}l@{}}Spoofax's code completion services do not\\ yet work with all languages and are in\\ process of a rewrite. As such, code completion\\ has not yet been attempted.\end{tabular}\\ \hline
\begin{tabular}[c]{@{}l@{}}Hover over variables to see\\ value, type and others\end{tabular} & No            & \begin{tabular}[c]{@{}l@{}}Due to limitations in time and the fact that\\ the environment is captured in a variable\\ of type \texttt{Object}, this feature has not been\\ implemented.\end{tabular}\\ \hline
Literate programming                                                                         & No            & \begin{tabular}[c]{@{}l@{}}An attempt has been made at writing an\\ IPython frontend for the REPL backend.\\ However, due to limitations in time and the\\ pressure of finishing more important features,\\ this work has not been finished.\end{tabular} \\ \hline
\begin{tabular}[c]{@{}l@{}}Integration with other\\ IDEs (IntelliJ)\end{tabular}             & No            & \begin{tabular}[c]{@{}l@{}}The Eclipse frontend is not completely done\\ yet, which was a requirement before starting\\ on implementing a frontend for IntelliJ. As such,\\ this feature has not been attempted.\end{tabular}\\ \hline
\end{tabular}
\caption{Evaluation of the requirements, separated per MoSCoW category}
\label{table:requirements}
\end{table}

\subsection{Evaluation of the Design Goals}
\label{ssec:eval-design-goals}

In this subsection the design goals as set out in \cref{ssec:goals} are
discussed. Each design goal is listed below, along with an explanation as to how
it has been met.

\paragraph{Language-agnostic} This design goal stated that no assumptions should
be made on the host language. The REPL should work for any language, without the
language designer having to configure it. However, should the language designer
want additional commands or language-specific configuration of the REPL, this
should be possible.

Both the frontends and the backend place no assumptions on the language:
everything automatically works for any language defined in Spoofax. If wanted,
the language designer can specify REPL-specific configuration in an esv file.
Currently, however, only DynSem-based interpreters are supported, as opposed to
Java- and Stratego-based interpreters. Despite this, care has been taken to make
this easily extendable. This design goal, therefore, is met.

\paragraph{Autogeneration} This design goal stated that the REPL should
automatically be provided, just like all of Spoofax's other services.

The REPL exists automatically for all languages, due to it reusing the existing
services offered by Spoofax. This means that it is automatically provided,
without any effort at the hands of the language designer or -user. Therefore,
this design goal is met.

% TODO: niet echt tevreden met deze paragraaf. Suggesties?
\paragraph{Maintainability} This design goal stated that the code should be
maintainable, because maintainership will be transferred to the people already
working on Spoofax.

Various static analysis tools, up-to-date documentation, proper software design
and a received 4,5 out of 5 stars during the first round of feedback from the
Software Improvement Group (see \cref{ssec:sig} below), are good indications that
this design goal is met.

\paragraph{IDE-agnostic} This design goal stated that the REPL should place no
assumptions on the environment it runs in, since an effort is underway to do the
same for Spoofax itself.

The backend is agnostic to any frontend, no assumptions are made whatsoever:
console or graphical, blocking or unblocking user input, single-threaded or
multi-threaded, et cetera. This design goals is met.

\paragraph{Performance} This design goal stated that the REPL should be
performant, because care is taken to ensure that Spoofax itself is performant.

Launching the REPL is near instantaneous. Commands and expressions are evaluated
in real time, without noticable delay. This design goals is thus met.

\paragraph{Modify Spoofax's existing codebase as little as possible} This design
goal stated that the REPL should be an extension to Spoofax, which means that
the existing codebase should be modified as little as possible.

The REPL is developed into a standalone repository. It uses the same API exposed
by Spoofax as any other extension would use. Minimal changes have been made to
Spoofax itself: the only large change that went upstream is the ability to
configure the REPL in an esv file. However, DynSem had to be significantly
modified after its rewrite to support the use case of a REPL. These changes were
made in cooperation with its maintainer, and most were features that had to be
implemented regardless. As such, this design goal is also met.

\subsection{Evaluation of the Product}
\label{ssec:eval-product}

This subsection evaluates whether the minimal viable product, as defined in
\cref{ssec:mvp}, has been delivered. The minimal viable product has been defined
as all the ``must have'' requirements (see \cref{ssec:requirements}), including
the design goals of being language-agnostic and having the REPL being generated
automatically.

The previous section concluded that all the design goals have been met.
\Cref{ssec:eval-requirements} and \cref{table:requirements} show that all the
``must have'' requirements have been implemented, except one: syntax
highlighting.

% TODO: finish this section.

\section{Process Evaluation}
\label{sec:process-evaluation}

This section gives an evaluation of the process by which the delivered product
came to be. First, the methodologies used during development are discussed in
\cref{ssec:dev-meth}. \Cref{ssec:sig} shows the feedback received from the
Software Improvement Group and discusses the changes that were made to the code
accordingly.

\subsection{Development Methodologies}
\label{ssec:dev-meth}

In the beginning of the project, agreements were made on the development
methodologies that were to be used. Throughout the duration of the project, some
of these methodologies turned out to work well, whilst others were less
effective than anticipated. This section discusses these agreements.

The Scrum methodology was chosen to manage the product development. Previous
experience showed that one week sprints are often too short, so for this project
biweekly sprints were decided. This worked well: there was enough time to make
informed decisions and to evaluate or rework certain things. Sprints were
managed flexibly: if the sprint plan needed to change due to unexpected issues,
biweekly sprints allow for an easier revision of the sprint plan.

Sprints were evaluated with a meeting with the client, during which the progress
was demoed. The feedback received during these meetings was valuable and taken
into account to guide the next sprint.

The client mandated the use of Trello to manage the sprints, which has been a
positive experience as well. Trello makes it easy to manage deadlines and to
know what has to be done at what time. Managing a backlog of user stories and
moving them around from, for example, ``developing'' to ``testing'', is
straightforward and helps to keep everyone, including the client, updated on the
progress.
The pull-based development model worked well, as was anticipated from previous
experience. The continuous integration, in combination with a set of strict
static analysis tools, ensured that the code review process had a positive
influence on the quality of the code.\\

Naturally, there were also things that could have gone better. It is easy to
underestimate the amount of work when putting together a sprint plan, which has
been a source of problems. The solution for this issue is to deliberately plan
less: it is easier to move new user stories to the sprint plan than it is to
decide which user stories not to complete.

A second issue concerns the unstable development environment provided by Eclipse.
Omitting all the details, several hard to track issues occurred which took a
considerable amount of time to resolve. To add to this, on several occurrences
the development of the Eclipse plugin had to wait for changes to make it into
Spoofax.

The biggest issue that occurred during the development is the fact that Spoofax
is a moving target. Halfway through the allotted time for development, DynSem
was significantly rewritten. After this rewrite, the functionality required for
a REPL was no longer present. To resolve this, considerable effort has been spent
to add these features to DynSem in cooperation with DynSem's maintainer. All of
this work has been accepted upstream and has now been published in the latest
snapshot releases.\\

To conclude, the process by which the product came to be went predominantly
well, despite some issues that can be improved during future projects.

\subsection{Software Improvement Group}
\label{ssec:sig}

The code was sent to the Software Improvement
Group\footnote{\url{https://www.sig.eu/}} for review on May 27th. The following
feedback was received on the third of June:

\begin{quotation}
[Analyse]

De code van het systeem scoort 4,5 ster op ons onderhoudbaarheidsmodel, wat
betekent dat de code bovengemiddeld onderhoudbaar is. De hoogste score is niet
behaald door een lagere score voor Unit Interfacing.

Voor Unit Interfacing wordt er gekeken naar het percentage code in units met een
bovengemiddeld aantal parameters. Doorgaans duidt een bovengemiddeld aantal
parameters op een gebrek aan abstractie. Daarnaast leidt een groot aantal
parameters nogal eens tot verwarring in het aanroepen van de methode en in de
meeste gevallen ook tot langere en complexere methoden.

In jullie geval heeft de constructor van \texttt{Data} een enorm aantal
parameters. Er zijn grofweg twee manieren om dat te addresseren: de eerste is
het groeperen van velden in sub-types. Een flauw voorbeeld is het groeperen van
\texttt{``width''} en \texttt{``height''} in een type \texttt{``Dimensions''}.
Dit voorbeeld is natuurlijk triviaal, maar je ziet wel vaak dat studenten aan
het begin wat huiverig zijn om kleine classes voor alleen een type te
introduceren, terwijl dit op de lange termijn vaak wel degelijk zin heeft. De
tweede manier is niet meer alle parameters in de constructor meegeven, maar via
setter-methodes. Dat leidt tot meer code, maar maakt het intialiseren van een
object wel leesbaarder.

Een ander voorbeeld van dezelfde situatie is de constructor van
\texttt{TransformCommand} (zo te zien krijgen jullie daar ook al een
waarschuwing vanuit CheckStyle). Dit geval is wat moeilijker op te lossen,
aangezien jullie hier van dependency injection gebruik maken en de methode
daardoor wat moeilijker kunnen refactoren.

Het is goed om te zien dat jullie naast productiecode ook veel testcode hebben
geschreven. De verhouding tussen de hoeveelheid productiecode en testcode is met
bijna 1 op 1 ook zeer gezond. Hopelijk lukt het jullie om dat vast te houden
tijdens het vervolg van het project.
\end{quotation}

This analysis says that the code scored bad on unit interfacing, because there
were quite a few classes that had an above average amount of constructor
parameters. The reason that this came to be is the fact that dependency
injection is used, as opposed to the traditional way of managing dependencies.
This fact is also mentioned in the analysis itself.

Two suggestions were given as a possible solution to this problem: create new
classes that encapsulate several parameters, or use \texttt{``setter''} methods
to pass parameters this way. However, neither solution was satisfactory as-is:
data classes are often considered to be a code smell, and \texttt{setter}
methods risk not fully initialising an object. The classes to which the
criticism applied have been refactored into smaller, more managable units.
Ultimately, this not only led to resolving the only negative feedback made by
the Software Improvement Group, it also led to a better architecture.

%%% Local Variables:
%%% mode: latex
%%% TeX-master: "main"
%%% End:
