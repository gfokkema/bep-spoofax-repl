\chapter{Project Description}
\label{cha:project-description}
\addtocontents{toc}{\protect\setcounter{tocdepth}{0}}

\paragraph{Create a command-line shell interface for the Spoofax Language
Workbench}\mbox{}\\

From Spoofax's website:

\begin{quotation}
Spoofax is a platform for developing textual domain-specific languages with
full-featured Eclipse editor plugins.
\end{quotation}

A feature that Spoofax is lacking is a Read-Eval-Print Loop (REPL) service. A
REPL is an interactive programming environment that takes single expressions,
evaluates them and prints the result(s). REPLs are a popular tool for
programming because they facilitate exploratory programming and debugging.
Common examples include command-line shells such as Bash and Python's REPL.

The deliverable for this project, then, is to create such a REPL for the Spoofax
Language Workbench.

%%% Local Variables:
%%% mode: latex
%%% TeX-master: "main"
%%% End:
